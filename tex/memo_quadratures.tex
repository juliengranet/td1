\documentclass[a4paper,11pt,twoside]{article}

\usepackage[french]{babel}  % Pour le français
\usepackage[utf8]{inputenc} % Pour taper les caractères accentués
\usepackage[T1]{fontenc}

\usepackage{amsmath}  % Ces trois paquets donnent accès à 
\usepackage{amsfonts} % des symboles et formulations
\usepackage{amstext}  % mathématiques
\usepackage{hyperref} % Permet de faire automatiquement des liens dans les
                      % documents
\usepackage{booktabs}
\usepackage{graphicx} % Permet d'insérer des images
\DeclareGraphicsExtensions{.png} % Mettre ici la liste des extensions des
                                 % fichiers images

% On peut choisir la police en utilisant un paquet 
%\usepackage{newcent}
\usepackage{lmodern}
%\usepackage{cmbright} % Computer Modern Bright

% Une des nombreuses manière de modifier les marges par défaut
\usepackage{geometry}
\geometry{vmargin=2cm,hmargin=2.5cm,nohead}

% On peut redéfinir certaines longueurs, par exemple l'espacement entre
% les paragraphes -- ce n'est pas une bonne idée, d'un point de vue
% typographique.
% \setlength{\parskip}{0.25cm}

% Quelques définitions 
\def \rr {{\mathbb R}} % L'ensemble R
\def \cc {{\mathbb C}} % L'ensemble C
\def \nn {{\mathbb N}} % L'ensemble N
\def \zz {{\mathbb Z}} % L'ensemble Z

% Les informations de la page de titre (page de titre séparée pour un 'report').
\title{Brefs rappels de formules de quadrature}
\author{
  Yves Coudière\\\href{mailto:yves.coudiere@u-bordeaux.fr}{\small\tt yves.coudiere@u-bordeaux.fr}
}
\date{\today} % \today pour la date courante 

\begin{document}

\maketitle % Page de titre automatique à partir des infos ci-dessus

\subsection*{Formules élémentaires}
Ce sont les formules de quadrature, notées génériquement $Qf$ d'une
fonction $f$ définie sur un intervalle de référence
$[\alpha,\beta]$. Ces formules servent à calculer une approximation de
l'intégrale $If = \int_\alpha^\beta f(t) dt$. Elles ont la forme
générale $Qf = \sum_{k=1}^m \omega_kf(t_k)$ où $\omega_k\in\rr$ et
$\alpha\le t_k \le \beta$ sont les poids et les points
d'interpolation. Voir des exemples dans le tableau~\ref{tab:quad}.
\begin{table}
  \centering
  \begin{tabular}{llllll}\toprule
    Formule & $\alpha,\beta$ & $m$ & $\omega_i$ & $t_i$ & $p$ \\
    \midrule
    trapèzes     &  0,1 & 2 & $\frac12,\frac12$ & 0,1 & 1 \\
    Simpson      & -1,1 & 3 & $\frac13,\frac43,\frac13$ & -1,0,1 & 3
    \\
    point milieu & -1,1 & 1 & 2                   & 0   & 1 \\
    Gauss-Legendre 2 & -1,1 & 2 & 1,1 & $-\frac{\sqrt{3}}3,
                                        -\frac{\sqrt{3}}3$ & 3 \\
    Gauss-Legendre 3 & -1,1 & 3 & $\frac59,\frac89,\frac59$ &
                                                                  $-\frac{\sqrt{15}}5,0,\frac{\sqrt{15}}5$
                                                        & 5 
    \\ \bottomrule
  \end{tabular}
  \caption{Quadratures classiques.}
  \label{tab:quad}
\end{table}

Ces formules sont caractérisées par le degré maximal $p$ des polynomes
intégrés exactement, c'est à dire pour lesquels on a $If = Qf$.


\subsection*{Formules composées}
Ce sont les formules obtenues en composant les formules élémentaires sur
un découpage de l'intervalle $[a,b]$ en $n$ sous-intervalles de même
longueur $h = \frac{b-a}n$. On procède de la manière suivante (découpage
de $[a,b]$ en $n$ sous-intervalles, puis changement de variable
$x = a+ih+h\frac{t-\alpha}{\beta-\alpha}$ pour ramener
$x \in [a+ih, a+(i+1)h]$ sur $t \in [\alpha,\beta]$ -- dans chaque
sous-intervalle):
\begin{multline*}
  \int_a^b f(x)dx = \sum_{i=0}^{n-1} \int_{a+ih}^{a+(i+1)h} f(x)dx =
  \sum_{i=0}^{n-1} \frac{h}{\beta-\alpha} \int_\alpha^\beta f\left(
    a+ih+h\frac{t-\alpha}{\beta-\alpha} \right) dt \\ =
  \frac{h}{\beta-\alpha} \sum_{i=0}^{n-1} \int_\alpha^\beta g_i(t)dt,
\end{multline*}
où $g_i(t) = f\left( a+ih+h\frac{t-\alpha}{\beta-\alpha} \right)$. On
applique alors une formule de quadrature élémentaire à la fonction $g_i$
sur l'intervalle de référence $[\alpha,\beta]$, et l'on trouve la
formule composée
\begin{equation*}
  Q_nf = \frac{h}{\beta-\alpha} \sum_{i=0}^{n-1} \sum_{k=1}^m \omega_k
  f\left( a+ih+h\frac{t_k-\alpha}{\beta-\alpha} \right).
\end{equation*}
On trouve par exemple,
\begin{itemize}
\item pour la formule des trapèzes,
  $Q_nf = h\sum_{i=0}^{n-1} \frac12\left( f(a+ih)+f(a+(i+1)h) \right)$,
\item pour la formule du point milieu,
  $Q_nf = \frac{h}2 \sum_{i=0}^{n-1} 2f(a+(i+0.5)h)$,
\item pour la formule de Simpson,

  $Q_nf = \frac{h}2 \sum_{i=0}^{n-1} \frac13\left( f(a+ih) +
    4f(a+(i+0.5h)) + f(a+(i+1)h) \right)$,
\item etc.
\end{itemize}

\end{document}
